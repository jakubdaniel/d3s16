\documentclass[d3s]{beamer}
\def\event{{D3S Seminar}}
%\def\date{{2014-09-23}}
%\usetheme[event=\event,footlineauthor]{D3S}
\usetheme[event=\event]{D3S}
\usepackage{colortbl}
\title{Checking LTL Properties of Infinite-State Systems via Reduction to Safety}
\def\fbk{*}
\author{\vskip0.5em{\bfseries Jakub Daniel\phantom{\fbk}}, {Alberto Griggio\fbk}, {Stefano Tonetta\fbk}, {Alessandro Cimatti\fbk}, \vskip1.5cm \fbk Fondazione Bruno Kessler}
\begin{document}
\titleframe
\begin{frame}
\frametitle{Infinite-State Systems and Properties}
$\text{system} \sim (I, T)$ \\
example: \\
$$
\begin{array}{rcl}
I &\equiv& \big[ x = 0 \big] \\
T &\equiv& \big[ x < 10 \wedge x' = x + 1 \big] \vee \big[ x' = x \big] \\
P &\equiv& G\ x \le 10 \\
\end{array}
$$
\vskip1em
\centering\includegraphics{slides7}
\end{frame}
\begin{frame}
\frametitle{Reduction to Fairness}
\begin{itemize}
\item standard reduction to $FG\ \neg f$ (emptiness of $L(A_S \times A_{\neg \varphi})$)
\item counterexample is path where $GF\ f$
\item how to prove absence of such cex?
\end{itemize}
\end{frame}
\begin{frame}
\frametitle<1-2>{Insight from Termination Analysis}
\frametitle<3>{\ldots in Our Setting}
\frametitle<4>{Why \ldots}
\only<1-3>{Finite set $\Phi$ of well-founded binary relations to show}
\only<1-2>{finitness of chains of states in any computation}
\only<3>{$FG\ \neg f$ \\ \strut}
\only<4>{$FG\ \neg f$ iff $\exists \text{ finite set } \Phi \text{ of well-founded relations}$ such that \\ \strut}
\vskip1cm
\begin{eqnarray}
\only<3>{\bgroup\color{red}}T\only<3>{_f}^+\only<3>\egroup \cap (Acc \times Acc) \subseteq \bigcup \Phi \label{eqn:wf}
\end{eqnarray}
\vskip1cm
\only<1>{\strut \\}
\only<2-3>{\strut well-founded $\sim$ every non-empty set has a minimal element \\}
\only<4>{\strut Consider a path where $GF\ f$ and (\ref{eqn:wf}) holds for some $\Phi$. \\}
\end{frame}
\subtitleframe{\includegraphics{slides1}}
\subtitleframe{\includegraphics{slides2}}
\subtitleframe{\includegraphics{slides3}}
\subtitleframe{\includegraphics{slides4}}
\subtitleframe{\includegraphics{slides5}}
\begin{frame}
\frametitle{Contradiction}
\begin{itemize}
\itemsep=0.25cm
\item $\text{Path } \pi \models GF\ f$
\item \ldots
\item Infinite complete graph edge-labelled with $k$ colors
\itemsep=1cm
\item $\rightarrow$ infinite monochromatic clique (Ramsey's Theorem)
\itemsep=0.25cm
\item $\rightarrow$ infinite monochromatic $\Phi_i$-connected chain of states
\item $\rightarrow$ covering relation $\Phi_i$ is not well-founded
\item $\rightarrow$ contradiction
\end{itemize}
\end{frame}
\subtitleframe{}
\begin{frame}
\frametitle{Reduction to Safety}
Checking coverage of transitions $\in T^+ \cap (Acc \times Acc)$ with \emph{fixed} $\Phi$ is \\
\vskip1cm
\centering \Huge Safety Problem
\end{frame}
\begin{frame}
\frametitle{Discovery of $\Phi$}
Start from $\emptyset$ and build it up by learning from counterexamples. \\
\vskip0.5cm
\hbox to \hsize{\hfill\includegraphics{slides8}\hfill}
\vskip0.5cm
Incomplete but sound: Use conservative predicate abstraction to find an abstract lasso.\\
\vskip0.5cm
\centering\includegraphics{slides6}
\end{frame}
\begin{frame}
\frametitle{Refinement}
Two axes of refinement:
\begin{itemize}
\item abstraction: interpolants from infeasible finite unrolling
\begin{itemize}
\item fixed unrolling encoded as an unsatisfiable CNF formula
\end{itemize}
\item $\Phi$: termination argument (Farkas' lemma for reals and integers)
\begin{itemize}
\item lasso encoded as a conjunction of linear inequalities
\end{itemize}
\end{itemize}
\end{frame}
\begin{frame}
\frametitle{Possible Outcomes}
\begin{itemize}
\item Safe: property holds
\item Abstraction can be refined
\item Lasso can be blocked by augmenting $\Phi$ with wf relation
\item Cover counterexamples shorter than $n$ in hope of progress
\item Unknown
\end{itemize}
\end{frame}
\subtitleframe{}
\begin{frame}
\frametitle{Falsification}
\begin{itemize}
\item So far: searching for poof
\item Falsify by search for a concrete lasso (incomplete, use BMC)
\end{itemize}
\end{frame}
\subtitleframe{}
\begin{frame}
\frametitle{Implementation}
\begin{itemize}
\item IC3 with implicit abstraction as underlying safety checker
\begin{itemize}
\item Our encoding can be refined incrementally
\item IC3 in turn exploits incrementality of SMT
\item Everything done incrementally
\item Everything done symbolically (no explicit automata subtractions)
\end{itemize}
\end{itemize}
\end{frame}
\subtitleframe{}
\begin{frame}
\frametitle{Early Evaluation}
\centering
\def\us{\cellcolor{blue!100}}
\def\na{\color{darkgray}$^\text{\tiny N}\!/\!_\text{\tiny A}$}
\def\sum{\cellcolor{gray}}
\begin{tabular}{lr | r r r r r}
\multicolumn{2}{c|}{benchmark} & \multicolumn{5}{c}{solver} \\
\multicolumn{1}{l}{family} & \multicolumn{1}{l|}{\#} & \multicolumn{1}{l}{\us ic3ia} & \multicolumn{1}{l}{hsf} & \multicolumn{1}{l}{Ultimate} & \multicolumn{1}{l}{T2-CTL*} & \multicolumn{1}{l}{T2-term} \\
\sum all & \sum 807 & \us 562 & \sum 84 & \sum 299 & \sum 75 & \sum \na \\
bip          & 240 & \us 213 & 13 &  40 &  0 & \na \\
timed-toint  & 316 & \us 162 &  8 &  80 &  0 & \na \\
crafted      &  17 & \us  16 &  8 &   6 &  4 & \na \\
ultimate-ltl &  41 & \us  33 &  3 &  27 &  5 & \na \\
t2-testsuite & 193 & \us 138 & 52 & 146 & 66 & 191 \\
\end{tabular} \\
\vskip1cm
Submission to CAV'16
\end{frame}
\end{document}
